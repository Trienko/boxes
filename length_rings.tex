\documentclass[a4paper,10pt]{article}
\usepackage[utf8]{inputenc}
\usepackage{amsmath}
\usepackage[english]{babel}
\usepackage{amsthm}
\newtheorem{theorem}{Theorem}

%opening
\title{Last two layers of $G_n$}
\author{}

\begin{document}

\maketitle

%\begin{abstract}

%\end{abstract}
We assume that the reader has a general background in the combinatorics on words. An alphabet is a finite set $\Sigma$ of symbols. For an alphabet of $n$ symbols, we write $\Sigma_n$. A word $w$ or string $s$ is any sequence of letters from $\Sigma$. $\Sigma^{*}$ is the set of all words over $\Sigma$.

Let $G_n$ be the innerbox graph defined in our previous paper. The nodes of this graph consist of all the innerboxes contained in $\Sigma_n^*$. Two nodes are only connected via an edge if the word formed by merging the innerboxes associated with the two nodes have size $n+2$. If we only look at the last two layers of $G_n$ we immediatly notice that they are of the same size $n!$, i.e. they are of the same size as the symmetric group $S_n$. We will now show that these last two layers can be constructed using three types of cycles. In the last layer of $G_n$ we have the cycles formed by chaining together the innerboxes of length $n+1$. In the second last layer we have the cycles which can be formed by chaining together all the innerboxes of length $n$. Then we can connect the two layers via cycles that are constructed by intermittently chaining together one innerbox of length $n$ and one innerbox of length $n+1$.  


\section{Long Cycles}
Let us first investigate the cycles in the last layer of $G_n$. We find the subsequent innerbox $b_2$ which follows an innebox $b_1$ (both of length $n+1$) by copying the last $n$ digits of $b_1$ and appending to this intermediate word the second symbol found in $b_1$.   Let us now, without the loss of generality consider the following $n+1$ innerbox pairs:

\begin{equation}
010\rightarrow101:a_2=\begin{pmatrix}
0&1\\
1&0
\end{pmatrix} = (01)
\end{equation}
\begin{equation}
0120\rightarrow1201:a_3=\begin{pmatrix}
0&1&2\\
1&2&0
\end{pmatrix} = (012)
\end{equation}
\begin{equation}
01230\rightarrow12301:a_4=\begin{pmatrix}
0&1&2&3\\
1&2&3&0
\end{pmatrix} = (0123)
\end{equation}

From the above equations we see that we can also calculate the subsequent innerbox $b_2$ of size $n+1$ which follows an innerbox $b_1$ of size $n+1$ via a permutation $a_n$. Continuing along these lines we find that:

\begin{equation}
a_n = \overbrace{(0123\cdots n-1)}^n 
\end{equation}

The permutation which generates one innerbox from another innerbox of size $n+1$ will always have the same cycle type as it comes about via the same geometric process. Furthermore, as all permutations are present in the first $n$ characters of the innerboxes of size $n+1$ we will obtain all of the cycle permutations that is of the same cycle type as $a_n$ if we consider all innerbox pairs of length $n+1$ that connect with one another.

Moreover, we see that $a_n$ is a cyclic permutation of length $n$, i.e $|\!\!<\!\!a_n\!\!>\!\!|=n$. The size of the innerbox cycles in the last layer of $G_n$ is therefore $n$. As an example:

\begin{equation}
0120\rightarrow 1201 \rightarrow 2012 
\end{equation}


\section{Short Cycles}
Similarly we can consider the cycles in the second last layer of $G_n$. We can construct the subsequent innerbox $b_2$ of length $n$ following an innerbox $b_1$ of length $n$ by copying the last $n-2$ symbols from $b_1$ and appending to this intermediate word the alphabet symbol in $\Sigma_n$ that is not present in $b_1$ following the third symbol of $b_1$. Note there is one exception when $n=2$. In this case we simply replace all the symbols in $b_1$ (which is the same symbol) with the symbol in $\Sigma$ not contained in $b_1$. Let us now consider without any loss in generality the following innerbox pairs:
\begin{equation}
00\rightarrow11:b_2=
\begin{pmatrix}
0&1\\
1&0
\end{pmatrix} = (01)
\end{equation}
\begin{equation}
010\rightarrow020:b_3=
\begin{pmatrix}
0&1&2\\
0&2&1
\end{pmatrix} = (0)(12)
\end{equation}
\begin{equation}
0120\rightarrow2032:b_4=
\begin{pmatrix}
0&1&2&3\\
2&0&3&1
\end{pmatrix} = (0231)
\end{equation}
\begin{equation}
01230\rightarrow23042:b_5=
\begin{pmatrix}
0&1&2&3&4\\
2&3&0&4&1
\end{pmatrix} = (02)(134)
\end{equation}
\begin{equation}
012340\rightarrow234052:b_6=
\begin{pmatrix}
0&1&2&3&4&5\\
2&3&4&0&5&1
\end{pmatrix} = (024513)
\end{equation}
\begin{equation}
0123450\rightarrow2345062:b_7=
\begin{pmatrix}
0&1&2&3&4&5&6\\
2&3&4&5&0&6&1
\end{pmatrix} = (024)(1356)
\end{equation}
Continuing along these lines we obtain:

\begin{equation}
b_n = \left\{
\begin{array}{ll}
\overbrace{(02\cdots[n-2][n-1]13\cdots[n-3])}^{n}&\textrm{if}~n~\textrm{mod}~2=0\\
\overbrace{(02\cdots[n-3])}^{\frac{n-1}{2}}\overbrace{(13\cdots[n-2][n-1])}^{\frac{n+1}{2}}&\textrm{if}~n~\textrm{mod}~2=1 
\end{array}
\right.
\end{equation}

As was the case for $a_n$, since the generation algorithm used is geometric in nature the permutation which maps one innerbox of length $n$ to another innerbox of lenght $n$ will always be of the same cycle type. Note that, all the  permutations of $n$ can be constructed from the innerboxes of lenght $n$ by simply replacing the last symbol of the innerbox with the symbol in $\Sigma_n$ it does not already contain. This observation implies that we will obtain all permutations that have the same cycle type as $b_n$ if we consider all the possible innerbox pairs of length $n$. Furthermore,

\begin{equation}
|\!\!<\!\!b_n\!\!>\!\!|= \left\{
\begin{array}{ll}
n &\textrm{if}~n~\textrm{mod}~2=0\\
\textrm{lcm}(\frac{n-1}{2},\frac{n+1}{2}) &\textrm{if}~n~\textrm{mod}~2=1 
\end{array}
\right.
\end{equation}


% \begin{equation}
% \overbrace{(02\cdots[n-2][n-1]13\cdots[n-3])}^{n}
% \end{equation}
% 
% For odd $n$:
% \begin{equation}
% \overbrace{(02\cdots[n-3])}^{\frac{n-1}{2}}\overbrace{(13\cdots[n-2][n-1])}^{\frac{n+1}{2}} 
% \end{equation}
 
\section{Color Cycles}
We can also create cycles by interlacing innerboxes of size $n$ and $n+1$. Finding the innerbox $b_2$ of size $n$ which follows an innerbox $b_1$ of size $n+1$ is achieved by copying the last $n-2$ symbols of $b_1$ and then appending to it the third symbol contained in $b_1$. Moreover, obtaining the innerbox $b_2$ of size $n+1$ which follows an innerbox $b_1$ of size $n$ is obtained by copying the last $n-2$ symbols in $b_1$ and appending to it the alphabet symbol in $\Sigma_n$ not contained in $b_1$ followed by the second symbol in $b_1$. The effect of this interlacing is that we in effect map innerboxes of size $n+1$ to innerboxes of size $n+1$ via an intermediary innerbox of size $n$. A similar argument can be made regarding the innerboxes of size $n$. Let us now consider without loss of generality the follwing innerbox pairs of size $n+1$ which were obtained via the procedure above and an intermediary innerbox of size $n$:
\begin{equation}
010\rightarrow010:c_1=
\begin{pmatrix}
0&1\\
0&1
\end{pmatrix} = (1)(0)
\end{equation}

\begin{equation}
0120\rightarrow0210:c_2=
\begin{pmatrix}
0&1&2\\
0&2&1
\end{pmatrix} = (0)(12)
\end{equation}

\begin{equation}
01230\rightarrow30213:c_3=
\begin{pmatrix}
0&1&2&3\\
3&0&2&1
\end{pmatrix} = (2)(031)
\end{equation}

\begin{equation}
012340\rightarrow340213:c_4=
\begin{pmatrix}
0&1&2&3&4\\
3&4&0&2&1
\end{pmatrix} = (14)(032)
\end{equation}

\begin{equation}
0123450\rightarrow3450213:c_5=
\begin{pmatrix}
0&1&2&3&4&5\\
3&4&5&0&2&1
\end{pmatrix} = (03)(1425)
\end{equation}

\begin{equation}
01234560\rightarrow34560213:c_6=
\begin{pmatrix}
0&1&2&3&4&5&6\\
3&4&5&6&0&2&1
\end{pmatrix} = (25)(03614)
\end{equation}

\begin{equation}c_7=
\begin{pmatrix}
0&1&2&3&4&5&6&7\\
3&4&5&6&7&0&2&1
\end{pmatrix} = (147)(03625)
\end{equation}

\begin{equation}
c_8=
\begin{pmatrix}
0&1&2&3&4&5&6&7&8\\
3&4&5&6&7&8&0&2&1
\end{pmatrix} = (036)(147258)
\end{equation}

\begin{equation}
c_9=
\begin{pmatrix}
0&1&2&3&4&5&6&7&8&9\\
3&4&5&6&7&8&9&0&2&1
\end{pmatrix} = (258)(1470369)
\end{equation}

Continuing along these lines we obtain:

\begin{equation}
c_n = \left\{
\begin{array}{ll}
\overbrace{(03\cdots[n-3])}^{\frac{n}{3}}\overbrace{(14\cdots[n-2]25\cdots[n-1])}^{\frac{2n}{3}}&\textrm{if}~n~\textrm{mod}~3=0\\
\overbrace{(25\cdots[n-2])}^{\frac{n-1}{3}}\overbrace{(14\cdots[n-3]03\cdots[n-1])}^{\frac{2n+1}{3}} &\textrm{if}~n~\textrm{mod}~3=1\\
\overbrace{(14\cdots[n-1])}^{\frac{n+1}{3}}\overbrace{(03\cdots[n-2]25\cdots[n-3])}^{\frac{2n-1}{3}} &\textrm{if}~n~\textrm{mod}~3=2 
\end{array}
\right.
\end{equation}

% If $n$ is divisible by 3:
% 
% \begin{equation}
% \overbrace{(03\cdots[n-3])}^{\frac{n}{3}}\overbrace{(14\cdots[n-2]25\cdots[n-1])}^{\frac{2n}{3}}
% \end{equation}
% 
% If $n$ is dived by 3, 2 remains:
% \begin{equation}
% \overbrace{(14\cdots[n-1])}^{\frac{n+1}{3}}\overbrace{(03\cdots[n-2]25\cdots[n-3])}^{\frac{2n-1}{3}} 
% \end{equation}
% 
% If $n$ is dived by 3, 1 remains:
% \begin{equation}
% \overbrace{(25\cdots[n-2])}^{\frac{n-1}{3}}\overbrace{(14\cdots[n-3]03\cdots[n-1])}^{\frac{2n+1}{3}} 
% \end{equation}

If we consider the innerboxes of size $n$ instead we obtain the exact same relation $c_n$. This notion warrants further explanation. Without any loss in generality assume that $b_1$ is equal to $01\cdots[n-1]0$. Now consider the sequence of innerboxes $b_1$, $b_2$, $b_3$ and $b_4$ generated via the procedure outlined in the beginning of this section. It now follows that $b_3 = c_n(b_1)$ and $b_4 = c_n(b_2)$. Moreover, $c_n$ maps the first $n-2$ characters of $b_1$ to the last $n-2$ characters of $b_2$. Similarly, $c_n$ maps the first $n-2$ characters of $b_2$ to the last $n-2$ characters of $b_3$. As was the case for both $a_n$ and $b_n$, as the generation procedure we use to generate a box of size $n+1$ from another $n+1$ box via a box of size $n$ (and vice versa) is geometric in nature the permutation which maps the two $n+1$ innerboxes to one another will be of the same cycle type. As we can generate all permutations from the $n+1$ innerboxes of $\Sigma_n^*$ we will obtain all the permutations that has the same cycle type as $c_n$ if we consider all pairs of innerboxes of size $n+1$ which map to one another via a third innerbox of length $n$. Moreover,

\begin{equation}
|\!\!<\!\!c_n\!\!>\!\!|= \left\{
\begin{array}{ll}
\textrm{lcm}(\frac{n}{3},\frac{2n}{3}) &\textrm{if}~n~\textrm{mod}~3=0\\
\textrm{lcm}(\frac{n-1}{3},\frac{2n+1}{3}) &\textrm{if}~n~\textrm{mod}~3=1\\
\textrm{lcm}(\frac{n+1}{3},\frac{2n-1}{3}) &\textrm{if}~n~\textrm{mod}~3=2
\end{array}
\right.
\end{equation}

\section{Conjugancy Classes}
Consider a group $G$. Recall that the following equivalence class that contains the element $a\in G$
\begin{equation}
 \textrm{Cl} (a)=\left\{gag^{-1}:g\in G\right\}
\end{equation}
is known as the conjugancy class of $a$. 

\begin{theorem}
The conjugacy class $\textrm{Cl}(x)$ of an element $x \in S_n$ consists of all the elements of $S_n$ whose cycle type is the same as the cycle type of $x$.  
\end{theorem}

\begin{proof}
TBA
\end{proof}

\begin{theorem}
The subgroup generated by the conjugacy class $\textrm{Cl}(x)$ of an element $x \in S_n$ is normal in $S_n$.   
\end{theorem}
\begin{proof}
Let $S = \textrm{Cl}(x)$ and $H = <S>$. For any $\sigma\in S_n$ we have that
\begin{equation}
\sigma H \sigma^{-1} = \sigma <S> \sigma^{-1} = <\sigma S \sigma^{-1}> = <S> = H 
\end{equation}
The third equation follows from the second since the conjugates of products are equal to the products of conjugates. The fourth equation follows from the fact that $S$ is a conjugacy class of $S_n$.
\end{proof}

\begin{theorem}
If $n \geq 5$ then the only normal subgroups of $S_n$ are the trivial subgroup, the alternating group $A_n$ and $S_n$ itself. 
\end{theorem}
\begin{proof}
TBA 
\end{proof}

\begin{theorem}
$\textrm{Cl}(a_n)$ and $\textrm{Cl}(c_n)$ generates $S_n$. 
\end{theorem}

\begin{theorem}
$\textrm{Cl}(b_n)$ and $\textrm{Cl}(c_n)$ generates $S_n$. 
\end{theorem}





























\end{document}
