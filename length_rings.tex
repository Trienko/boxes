\documentclass[a4paper,10pt]{article}
\usepackage[utf8]{inputenc}
\usepackage{amsmath}

%opening
\title{Length Cycles}
\author{}

\begin{document}

\maketitle

%\begin{abstract}

%\end{abstract}

\section{Long Cycles}
We have:

\begin{equation}
010\rightarrow101:\begin{pmatrix}
0&1\\
1&0
\end{pmatrix} = (01)
\end{equation}
\begin{equation}
0120\rightarrow1201:\begin{pmatrix}
0&1&2\\
1&2&0
\end{pmatrix} = (012)
\end{equation}
\begin{equation}
01230\rightarrow12301:\begin{pmatrix}
0&1&2&3\\
1&2&3&0
\end{pmatrix} = (0123)
\end{equation}

For $n$:
\begin{equation}
\overbrace{(0123\cdots n-1)}^n 
\end{equation}

\section{Short Cycles}
We have:
\begin{equation}
00\rightarrow11:
\begin{pmatrix}
0&1\\
1&0
\end{pmatrix} = (01)
\end{equation}
\begin{equation}
010\rightarrow020:
\begin{pmatrix}
0&1&2\\
0&2&1
\end{pmatrix} = (0)(12)
\end{equation}
\begin{equation}
0120\rightarrow2032:
\begin{pmatrix}
0&1&2&3\\
2&0&3&1
\end{pmatrix} = (0231)
\end{equation}
\begin{equation}
01230\rightarrow23042:
\begin{pmatrix}
0&1&2&3&4\\
2&3&0&4&1
\end{pmatrix} = (02)(134)
\end{equation}
\begin{equation}
012340\rightarrow234052:
\begin{pmatrix}
0&1&2&3&4&5\\
2&3&4&0&5&1
\end{pmatrix} = (024513)
\end{equation}
\begin{equation}
0123450\rightarrow2345062:
\begin{pmatrix}
0&1&2&3&4&5&6\\
2&3&4&5&0&6&1
\end{pmatrix} = (024)(1356)
\end{equation}

For even $n$:
\begin{equation}
\overbrace{(02\cdots[n-2][n-1]13\cdots[n-3])}^{n}
\end{equation}

For odd $n$:
\begin{equation}
\overbrace{(02\cdots[n-3])}^{\frac{n-1}{2}}\overbrace{(13\cdots[n-2][n-1])}^{\frac{n+1}{2}} 
\end{equation}

\section{Color Cycles}
We have:
\begin{equation}
010\rightarrow010:
\begin{pmatrix}
0&1\\
0&1
\end{pmatrix} = (1)(0)
\end{equation}

\begin{equation}
0120\rightarrow0210:
\begin{pmatrix}
0&1&2\\
0&2&1
\end{pmatrix} = (0)(12)
\end{equation}

\begin{equation}
01230\rightarrow30213:
\begin{pmatrix}
0&1&2&3\\
3&0&2&1
\end{pmatrix} = (2)(031)
\end{equation}

\begin{equation}
012340\rightarrow340213:
\begin{pmatrix}
0&1&2&3&4\\
3&4&0&2&1
\end{pmatrix} = (14)(032)
\end{equation}

\begin{equation}
0123450\rightarrow3450213:
\begin{pmatrix}
0&1&2&3&4&5\\
3&4&5&0&2&1
\end{pmatrix} = (03)(1425)
\end{equation}

\begin{equation}
01234560\rightarrow34560213:
\begin{pmatrix}
0&1&2&3&4&5&6\\
3&4&5&6&0&2&1
\end{pmatrix} = (25)(03614)
\end{equation}

\begin{equation}
\begin{pmatrix}
0&1&2&3&4&5&6&7\\
3&4&5&6&7&0&2&1
\end{pmatrix} = (147)(03625)
\end{equation}

\begin{equation}
\begin{pmatrix}
0&1&2&3&4&5&6&7&8\\
3&4&5&6&7&8&0&2&1
\end{pmatrix} = (036)(147258)
\end{equation}

\begin{equation}
\begin{pmatrix}
0&1&2&3&4&5&6&7&8&9\\
3&4&5&6&7&8&9&0&2&1
\end{pmatrix} = (258)(1470369)
\end{equation}

If $n$ is divisible by 3:

\begin{equation}
\overbrace{(03\cdots[n-3])}^{\frac{n}{3}}\overbrace{(14\cdots[n-2]25\cdots[n-1])}^{\frac{2n}{3}}
\end{equation}

If $n$ is dived by 3, 2 remains:
\begin{equation}
\overbrace{(14\cdots[n-1])}^{\frac{n+1}{3}}\overbrace{(03\cdots[n-2]25\cdots[n-3])}^{\frac{2n-1}{3}} 
\end{equation}

If $n$ is dived by 3, 1 remains:
\begin{equation}
\overbrace{(25\cdots[n-2])}^{\frac{n-1}{3}}\overbrace{(14\cdots[n-3]03\cdots[n-1])}^{\frac{2n+1}{3}} 
\end{equation}



















\end{document}
